\documentclass[11pt]{article}

% -------------------------
% Packages
% -------------------------
\usepackage[a4paper,margin=1in]{geometry}
\usepackage{amsmath,amssymb}
\usepackage{graphicx}
\usepackage{physics}
\usepackage{siunitx}
\usepackage{booktabs}
\usepackage{hyperref}
\usepackage{caption}
\usepackage{subcaption}

% -------------------------
% Title Information
% -------------------------
\title{\textbf{On the Misinterpretation of Wave Energy in the Assessment of Pile Loads}\\
\large A Comparison Between Energy-Based and Morison Force-Based Approaches}

\author{
Nikhil Chutturu\\
\small Mooring Engineer, Turnagain Marine Construction\\
}

\date{\today}

% -------------------------
\begin{document}
\maketitle

\begin{abstract}
Wave loading on vertical piles is sometimes interpreted through an energy-based
framework, by analogy with berthing or impact problems.
This technical note demonstrates that such an interpretation leads to
non-physical conclusions when applied to wave-induced pile response.
A direct comparison is presented between an energy-flux-based formulation
and a force-based formulation using Morison’s equation.
The analysis shows that wave-induced pile response is fundamentally governed
by hydrodynamic forces derived from particle kinematics rather than
by accumulated wave energy.
\end{abstract}

\vspace{1em}

% =========================================================
\section{Introduction}

Vertical piles are commonly used to support marine structures such as piers,
floating docks, and dolphins.
Design wave loads on such piles are typically evaluated using force-based
methods derived from linear wave theory and hydrodynamic coefficients.

Occasionally, wave loading is interpreted as an energy-transfer problem,
motivated by the use of energy concepts in berthing and fender design.
This note examines the validity of such an interpretation.

% =========================================================

% =========================================================
\section{Wave Dispersion and Group Velocity}

For linear waves, the dispersion relation is given by
\begin{equation}
\omega^2 = g k \tanh(kh)
\end{equation}
where $\omega = 2\pi/T$ is the angular frequency and $k$ is the wave number.

The phase velocity is
\begin{equation}
C = \frac{\omega}{k}
\end{equation}

and the corresponding group velocity is
\begin{equation}
C_g = \frac{1}{2} C
\left( 1 + \frac{2kh}{\sinh(2kh)} \right)
\end{equation}

The group velocity governs the propagation of wave energy.

% =========================================================
\section{Energy-Based Interpretation}

The mean wave energy per unit horizontal area is
\begin{equation}
E = \frac{1}{8} \rho g H^2
\end{equation}

The wave energy flux per unit crest width is therefore
\begin{equation}
P = E C_g
\end{equation}

If a pile of diameter $D$ is assumed to intercept this energy flux,
the corresponding power associated with the pile is
\begin{equation}
P_{\text{pile}} = P D
\end{equation}

Assuming continuous exposure, the accumulated energy over a duration $t$ is
\begin{equation}
E_{\text{acc}}(t) = P_{\text{pile}} t
\end{equation}

This formulation implies an unbounded increase in energy demand with time.

% =========================================================
\section{Force-Based Interpretation Using Morison Equation}

Wave loading on slender piles arises from hydrodynamic forces induced by
water particle velocity and acceleration.

The inline force per unit length is expressed by Morison’s equation:
\begin{equation}
f(z,t) =
\frac{1}{2} \rho C_D D u(z,t)\abs{u(z,t)}
+
\rho C_M \frac{\pi D^2}{4} a(z,t)
\end{equation}

where $u(z,t)$ and $a(z,t)$ are the horizontal particle velocity and
acceleration, respectively.

Under linear wave theory,
\begin{equation}
u(z,t) = \frac{\omega H}{2}
\frac{\cosh k(z+h)}{\sinh kh}
\cos(\omega t)
\end{equation}

\begin{equation}
a(z,t) = \omega u(z,t)
\end{equation}

The total inline force and overturning moment on the pile are obtained
by integrating along the pile length.

% =========================================================
\section{Comparison and Discussion}

The energy-based formulation predicts a load demand that increases
linearly with exposure duration, independent of structural response.
Such behavior is inconsistent with observed wave-induced pile behavior,
which remains oscillatory and bounded.

In contrast, the force-based Morison formulation produces finite,
physically meaningful forces and moments directly applicable
to structural analysis.

Wave energy is therefore best interpreted as a descriptor of sea state
severity rather than as a design load acting on piles.

% =========================================================
\section{Conclusions}

Wave-induced pile loading is fundamentally a force–response problem.
Energy-based interpretations, while intuitive, are not appropriate
for assessing pile strength or deflection under wave action.

Design methodologies should continue to rely on hydrodynamic force models
based on wave kinematics and established formulations such as Morison’s equation.

\vspace{1em}

\noindent\textbf{Keywords:} wave loading, piles, Morison equation, wave energy, coastal engineering

\begin{thebibliography}{9}

\bibitem{cem}
USACE, \emph{Coastal Engineering Manual}, EM 1110-2-1100.

\bibitem{morison}
Morison, J.R., O’Brien, M.P., Johnson, J.W., Schaaf, S.A.,
\emph{The Force Exerted by Surface Waves on Piles},
Petroleum Transactions, AIME, 1950.

\bibitem{sarpkaya}
Sarpkaya, T., Isaacson, M.,
\emph{Mechanics of Wave Forces on Offshore Structures},
Van Nostrand Reinhold, 1981.

\bibitem{dnv}
DNV-RP-C205,
\emph{Environmental Conditions and Environmental Loads}.

\end{thebibliography}


\end{document}
